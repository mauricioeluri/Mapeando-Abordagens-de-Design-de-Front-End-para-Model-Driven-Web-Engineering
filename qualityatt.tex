

\textbf{Critérios de Qualidade:} Os critérios de qualidade, importantes para avaliar a qualidade dos estudos selecionados, são mostrados na tabela~\ref{table:npapers}. Estes critérios são respondidos com um dos seguintes valores numéricos: Valor 1 - Completamente; Valor 0.5 - Parcialmente; Valor 0 - Não atende ao critério. 

\begin{table}[t!]
%\tiny
\small
%\scriptsize
  \caption{Critérios de Qualidade do Estudo}
  
    %\begin{tabular}{ | p{1.5cm} | p{15.2cm} |  p{1.7cm} |}%p{14.7cm} p{6.2cm}
		\begin{tabular}{ | p{0.5cm} | p{7.2cm} |}%p{14.7cm} p{6.2cm}
    \hline

     \textbf{Id}
    & \textbf{Título e Referência para o Estudo}
		\\ \hline

QC01 &  A avaliação da abordagem proposta segue um protocolo empírico? \\ \hline

QC02 &  A avaliação leva em conta contextos ágeis discutindo benefícios e limitações? \\ \hline

QC03 &  A abordagem foi amplamente testada em protótipos executáveis? \\ \hline

QC04 &  A abordagem é mencionada como adotada pela indústria de software?  \\ \hline

QC05 &  No workflow, além de uma DSL, utiliza-se de outras ferramentas além de geradores de código ou interpretadores? \\ \hline

%QC06 &  A DSL propõe algum diferencial com relação às demais? \\ \hline

%QC07 &  A DSL proposta é uma extensão ou nova versão de outra DSL? \\ \hline

%QC08 &  A DSL proposta possui meios de desenvolvimento colaborativos? \\ \hline

%QC09 &  A DSL tem mapeamento para geração de código em plataformas web no estado da arte? \\ \hline

\end{tabular}

\label{table:npapers} 
\end{table}



\begin{table*}[ht!]
%\tiny
\small
%\scriptsize
  \caption{Propostas para design, refinamento e geração de front-end web no MDWE} %13 years of toolboxes scoping the design of web front ends in MDWE}

\begin{tabular}{ | p{0.3cm} | p{10.0cm} | p{0.7cm} |  p{0.7cm} | p{0.7cm} | p{0.7cm} | p{0.7cm}| p{0.7cm}|}
    \hline

     \textbf{Id}
    & \textbf{Título e Referência para o Estudo}
    & \textbf{QC01}
    & \textbf{QC02}
     & \textbf{QC03}
      & \textbf{QC04}
       & \textbf{QC05}
        & \textbf{Ano}
		\\ \hline

S01 &  A mda-compliant environment for developing user interfaces of information systems~\cite{Vanderdonckt05} &1  &0 &0.5 &0.5 &1 & 2005 \\ \hline

S02 &  Rapid prototyping of web applications combining domain specific languages and model driven design~\cite{Nunes06}  &  & & & & & 2006 \\ \hline

S03 &  Automated prototyping of user interfaces based on uml scenarios~\cite{Elkoutbi06}  &  & & & & & 2006 \\ \hline

S04 &  A model-driven approach to generating user interfaces~\cite{Kavaldjian07}  &  & & & & & 2007 \\ \hline

S05 & A uml profile for modeling framework-based web information systems~\cite{Souza07}  &  & & & &  & 2007 \\ \hline

S06 &  Model-Based development of user interfaces a new paradigm in useware engineering~\cite{ZUEHLKE200731}  & 0 & 0.5 & 0 & 0 & 0 & 2007 \\ \hline		

S07 &  A component-and push-based architectural style for ajax applications~\cite{mesbah2008component}  &1  &0 & 0 &0.5 & 0  & 1 2008\\ \hline

S08 &  A model-driven development for gwt-based rich internet applications with ooh4ria~\cite{Melia08}  &  & & & &  & 2008 \\ \hline

S09 &  A language for high-level description of adaptive web systems~\cite{sadat2008language}  & 1  & 1 & 0.5 & 0 & 1  & 2008 \\ \hline

S10 &  Wysiwyg development of data driven web applications~\cite{Yang08}  &  & & & & & 2008 \\ \hline

S11 &  Oblivious integration of volatile functionality in web application interfaces~\cite{Ginzburg09}  &   & & & &  & 2009\\ \hline

S12 &  A model-driven approach to building modern semantic web-based user interfaces~\cite{chavarriaga2009model}  &  & & & &  & 2009 \\ \hline

S13 &  Model-driven development of composite context-aware web applications~\cite{KAPITSAKI20091244}  & 1 & 0.5 & 0 & 0 & 1  & 2009 \\ \hline		

%S14 &  Context modelling and a context-aware framework for pervasive service creation: A model-driven approach~\cite{ACHILLEOS2010281} & &	\\ \hline
%Tive que remover este pq não bate nos critérios de inclusão (não é para MDWE) e substituí pelo de baixo

S14 &  From Mockups to User Interface Models: An Extensible Model Driven Approach~\cite{Rivero10}  &  & & & & & 2010	\\ \hline 


S15 &  Generating blogs out of product catalogues: An mde approach~\cite{diaz2010generating}  & 1 & 0.5 & 0 & 0 & 0 & 2010 \\ \hline

S16 &  Transformation templates:  adding flexibility to model-driven engineering of user interfaces~\cite{Aquino10}  &  & & & &  & 2010 \\ \hline

S17 &  A DSL toolkit for deferring architectural decisions in DSL-based software design~\cite{ZDUN2010733}  &  & & & & & 2010 \\ \hline 

S18 & Specification of personalization in web application design~\cite{GARRIGOS2010991}	 &  & & & &  & 2010 \\ \hline	

S19 &  Using HCI Patterns within the Model-Based Development of Run-Time Adaptive User Interfaces~\cite{SEISSLER2010477}  &  & & & & & 2010 \\ \hline

S20 &  Building enterprise mashups~\cite{DEVRIEZE2011637}  &  & & & &  & 2011\\ \hline


S21 &  Towards model-driven development of access control policies for web applications~\cite{Busch12}   &  & & & &  & 2012 \\ \hline

S22 &  A framework for model-driven development of information systems: Technical decisions and lessons learned~\cite{Vara12}  &  & & & &  & 2012 \\ \hline

S23 &  Towards agile model-driven  web  engineering~\cite{Rivero12}  &  & & & & & 2012 \\ \hline

S24 & Ciat-gui:  A  mde-compliant  environment  for  developing  graphical  user interfaces of information systems~\cite{Molina12}  &  & & & &  & 2012 \\ \hline

S25 &  From requirements to web applications in an agile model-driven approach~\cite{Grigera12}  &  & & & &  & 2012 \\ \hline

S26 &  Visually modelling data intensive web applications to assist end-user development~\cite{Deufemia13}  &  & & & & & 2013 \\ \hline

S27 &  Teaching model driven engineering from a relational database perspective~\cite{Batory13MODELS}  &  & & & &  & 2013 \\ \hline

S28 &  Mockup-Driven Development: Providing agile support for Model-Driven Web Engineering~\cite{Rivero2014}  & 1 & 1 & 0.5 & 0 & 0 & 2014\\ \hline


S29 &  Seamless composition and reuse of customizable user interfaces with Spec~\cite{VANRYSEGHEM201434}	 &  & & & &  & 2014 \\ \hline

S30 &  Assisted tasks to generate pre-prototypes for web information systems~\cite{Basso14ICEISb}  & 0.5 & 1 & 1 & 0.5 & 0.5 & 2014 \\ \hline

S31 &  Large-scale model-driven engineering of web user interaction: The webml and webratio experience~\cite{Brambilla14}  &  & & & & & 2014 \\ \hline

S32 &  Combining mde and scrum on the rapid prototyping of web information systems~\cite{Basso15IJWET}  & 0.5 & 1 & 1 & 1 & 1 & 2015  \\ \hline

S33 &  A model-driven development for creating accessible web menus~\cite{antonelli2015model}  &1  &0 &0.5 & 0.5 & 1 & 2015 \\ \hline

S34 &  Automated design of multi-layered web information systems~\cite{Basso16JSS}  & 1 & 1 & 1 & 1 & 1  & 2016 \\ \hline

S35 &  An approach to build xml-based domain specific languages solutions for client-side web applications~\cite{chavarriaga2017approach}  & 1 & 0.5 & 0 & 0 & 0 & 2017 \\ \hline

%S36 &  Design annotations to improve API discoverability~\cite{SANTOS201717} & & \\ \hline
%Removi pq não entra no critério dei nclusão: MDWE

S36 &  DataMock: An Agile Approach for Building Data Models from User Interface Mockups~\cite{Rivero2017}  &  & & & & & 2017 \\ \hline
%Retirei da selecao

S37 &  Application of Kroki Mockup Tool to Implementation of Executable CERIF Specification~\cite{FILIPOVIC2017245}  & 1 & 0.5 & 0 & 0 & 0  & 2017 \\ \hline

S38 &  BRCode: An interpretive model-driven engineering approach for enterprise application~\cite{OLIVEIRA201886}  & 1 & 0 & 1 & 1 & 0 & 2018 \\ \hline

\end{tabular}

\label{table:npapers} 
\end{table*}
