===== Review =====


*** Number of papers published in previous SBSI editions cited (How many
related works published in previous editions of the SBSI Conference are
referenced (enter the quantity using numbers, i.e. 0 or 1, or n)? This
information is very important for statistical purposes, but it will not
be considered in the acceptance procedure.): 0

*** Comments to authors (Comments and suggestions to help the authors to
improve the quality of the paper (Please, provide theoretical and
methodological suggestions in a constructive way; suggest new references
and related works, and provide comments that justify the scores you
assigned in the items above). Please, consider suggesting the citation
of related papers previously published in SBSI editions, if
applicable.): O artigo propõe um estudo para apresentar abordagens para
MDWE.

Não há uma descrição precisa do que os autores entendem por MDWE. Há uma
citação para uma única referencia bibliográfica mas não é para Web
engineering especificamente. Da mesma forma o sentido de proposta de
abordagem para MDWE não é claramente citada pelos autores. Esperava-se
que o próprio conceito fosse objeto do levantamento do estudo.

A metodologia empregada, definida como os autores como um Estudo de
Mapeamento Sistemáticos está conflito como o que é defindo pelos autores
da referencia utilizada. Os autores citam Revisão Sistemática da
Literatura ou Mapeamento Sistemático. Estas duas metodologias são
distintas e tem objetivos distintos. A confusão entre os nomes não
permitem entender qual foi a adotada pelos autores.

As questões de pesquisa estão bastante abrangentes e necessitam de
questões anteriores. Antes de avaliar a evolução das abordagens, o
auxílio ao design, workflow de transformação e aplicação, os resultados
da revisão, o estudo deveria ter feito uma caracterização das abordagens
de MDWE considerado como os diferentes artigos definem cada uma delas e
quais são os processos e técnicas associadas.

Os resultados não respondem adequadamente às 4 questões de pesquisa,
pelo penos da forma como foi apresentada.


===== Review =====


*** Comments to authors (Comments and suggestions to help the authors to
improve the quality of the paper (Please, provide theoretical and
methodological suggestions in a constructive way; suggest new references
and related works, and provide comments that justify the scores you
assigned in the items above). Please, consider suggesting the citation
of related papers previously published in SBSI editions, if
applicable.): O artigo está mal sustentado. Não se percebe o motivo da
escolha do tópico. Também não é clara qual a noção de design para os
autores. Porque razão se fez o estudo num horizonte temporal de 13 anos?
Aconteceu algo de relevante em 2006? Não compreendo o uso da palavra
"mapeamento" no título. Isso significa exatamente o quê?

A frase "No desenvolvimento de sistemas de informações da web, os
front-ends da web como o layout são compostos pelos componentes para GUI
e por diagramas comportamentais" está imprecisa. Os fronts ends não são
compostos por diagramas... eu diria que as caraterísticas
(comportamentais ou estruturais) dos fronts ends podem ser modeladas em
diagramas. É difícil de se perceber a estrutura da string usada para
pesquisa do título. As palavras "modeling" e modelling" parecem estar
duplicadas. Não se percebe como o artigo "Automated prototyping of user
interfaces based on uml scenarios" foi incluído no estudo se as suas
palavras no título não incluem "web" OR "web-engineering" OR "website"
OR "websystem" OR "MDWE"? Dos 38 artigos S01-S38 só 20 (cerca de metade)
deles têm a palavra web no título, o que indicia que muitos deles
poderão estar fora do escopo do artigo.

O artigo falha ainda nos resultados apresentados. Por exemplo, a
resposta à RQ01 (evolução para suporte ao desenvolvimento ágil) parece 
estar focada na representação das camadas do MVC.

Não concordo com a frase "poucos programadores e equipes de
desenvolvimento realmente se preocupam em tornar seus websites
acessíveis em múltiplas plataformas". Esta parte em relaçao a UML
parece-me desnecessária já que é suposto ser de conhecimento geral: "uma
linguagem de modelagem de propósito geral usada em projetos
arquiteturais de alguns sistemas web." Esta frase parece óbvia: "isso
torna a construção do software dependente de tarefas de design." Não
percebo esta frase: "Um vez que os componentes de GUIs, estes podem ser
transformados para código ou refinados em modelos que tratam de outras
camadas além da de apresentação."

Considero este artigo pouco maduro para publicação. Precisa de uma
reescrita profunda.


===== Review =====


*** Comments to authors (Comments and suggestions to help the authors to
improve the quality of the paper (Please, provide theoretical and
methodological suggestions in a constructive way; suggest new references
and related works, and provide comments that justify the scores you
assigned in the items above). Please, consider suggesting the citation
of related papers previously published in SBSI editions, if
applicable.): Ótimo artigo, argumentos consistentes e organização de
ideias que realmente contribuem para a pesquisa na área de SI. Recomendo
a continuidade do estudo com a análise de novos conjuntos de assuntos
relacionados ao tema.